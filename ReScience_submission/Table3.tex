
\begin{table}[h!]
\begin{center}
 \begin{tabular}{||c | c | c c c ||} 
 \hline
   \multicolumn{2}{||c|}{Task 2 variant} & \multicolumn{3}{c||}{Deviation metric} \\ [0.5ex] 
\hline
No. of segments & Time steps & Mean & Median & Standard Deviation \\ [0.5ex] 
 \hline\hline
 3 & 10000 & 0.057 &	0.032 &	0.069 \\ 
 \hline
 4 & 10000 & 0.242 &	0.221 &	0.136 \\ 
 \hline
 5 & 10000 & 0.141 &	0.080 &	0.147 \\ 
 \hline
 6 & 10000 & 0.181 &	0.160 &	0.133 \\ 
 \hline
 7 & 10000 & 0.173 &	0.129 &	0.130 \\ 
 \hline
 8 & 10000 & 0.253 &	0.230 &	0.137 \\ 
 \hline
 9 & 10000 & 0.331 &	0.297 &	0.168 \\ 
 \hline
 10 & 10000 & 0.417 &	0.409 &	0.151 \\ 
 \hline
 15 & 10000 & 0.538 &	0.512 &	0.179 \\ 
 \hline 
 20 & 15000 & 0.366 &	0.324 &	0.188 \\ 
 \hline
 30 & 20000 & 0.549 &	0.489 &	0.236 \\ 
 \hline
 40 & 20000 & 0.372 &	0.313 &	0.228 \\ 
 \hline
 50 & 30000 & 0.375 &	0.283 &	0.281 \\ [1ex] 
 \hline
\end{tabular}
\end{center}
 \caption{Deviation metric showing the performance of the modified Python implementation on increasing number of segments for Task 2. Each variant is simulated with the default seed (5489) and ten additional seeds. The mean, median and standard deviation of the deviation metric over these eleven simulations are tabulated here.}
 \label{Table:deviation_scalability}
\end{table}
